\documentclass[../main.tex]{subfiles}

\begin{document}
\doublespacing
Mpemba effect was first discovered in $1963$ by a Tanzanian schoolboy Ernesto Mpemba while he was making ice cream. In a rush to get the space in freezer for his ice cream mixture, Mpemba decided not to let it cool down before placing it into the freezer. Surprisingly, his mixture froze before any other mixture despite having been placed shortly after it was boiled. Consequently, the effect was then described in a paper by professor Osborne \autocite{mpemba_cool_1969}, according to which it was initially defined and later on summarized by \textcite[514]{jeng_mpemba_2006} as follows:
\begin{quote}
    “If the two bodies of water, identical in every way, except that one is at a \textbf{higher temperature} than the other are exposed to identical \textbf{subzero conditions}, the initially hotter water will freeze first.”
\end{quote}
In recent years, Mpemba effect has seen a great increase in the number of plausible theories trying to offer an explanation behind its occurrence and in turn provide a deeper understanding of underlying physics at play. Even so, the effect still remains quite simple to describe but hard to predict and explain. In many respects, it is this mysterious and contradicting aspect of Mpemba effect that has intrigued dozens of scientists as well as inspired me to explore it further. \par

Among the numerous theories trying to offer potential explanations behind the Mpemba effect, \emph{Theory of Solutes} claims that the impact of solutes is what causes the initially hotter water to freeze faster than the cooler water. \textcite{katz_when_2009} suggested that the origin of the Mpemba effect was due to freezing-point depression by solutes, either gaseous or solid, whose solubility decreases with increasing temperature so that they are removed when water is heated. \par

In an attempt to explore the theory as well as the practical side, this work will turn to examining the original paper that first described the effect \autocite{mpemba_cool_1969}, study done by \textcite{katz_when_2009} and \textcite{thomas_mpemba_nodate}. In addition to this, work of \textcite{pankovic_mpemba_2012} will be considered. \par %Outline

\end{document}