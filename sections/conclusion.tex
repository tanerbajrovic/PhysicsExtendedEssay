\documentclass[../main.tex]{subfiles}

\begin{document}
\doublespacing

\subsection{Conclusion}

In this essay, I sought to explore and offer an answer to the following research question: (a) does hot water freeze faster than cooler water and (b) how do different concentrations of NaCl solute affect the occurrence of the Mpemba Effect. Results of the investigation imply that the cooler water reached \SI{0}{\celsius} before the initially hotter water for the control samples as seen in Table~\ref{tab:28Table}, Table~\ref{tab:50Table} and Fig.~\ref{fig:controlSamples}. This implies that the first hypothesis was not supported and it confirms the findings of \textcite{burridge_questioning_2016}. \par 
Further findings of the investigation implied that:  \par
(\textbf{i}) By increasing the mass concentration the elapsed time until \SI{0}{\celsius} decreased. This was observed for most samples on Table~\ref{tab:28Table} and Table~\ref{tab:50Table} with few anomalies that might be explained by the supercooling effect. Significant differences were observed between control samples and samples with $\SI{5}{\gram}$ of salt. Additionally, (\textbf{ii}) both samples shown on Fig.~\ref{fig:controlSamples} exhibited some unusual behavior around \SI{4}{\celsius} which can be attributed to water having the highest density. Lastly, (\textbf{iii}) rate of change of temperature for $\SI{125}{\milli\gram\per\milli\liter}$ samples was greater and thus resulted in a steeper temperature curve in comparison to the control. \par

Considering all of the previous conclusions, it should be noted that the Mpemba was not observed for any of the samples, although it was very close to being observed for the $\SI{125}{\milli\gram\per\milli\liter}$ samples in Fig.~\ref{fig:5GSamples}. This, nonetheless, implies that the experiment findings do not support the second hypothesis as well, in contrast to the studies conducted by \textcite{ibekwe_investigating_2016} and \textcite{katz_when_2009}, where the effect was observed.  \par

\end{document}